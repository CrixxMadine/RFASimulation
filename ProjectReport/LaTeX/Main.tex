%%%%%%%%%%%%%%%%%%%%%%%%%%%%%%%%%%%%%%%%%%%%%%%%%%%%%%%%%%%%%%%%%%
%
% Template für die Erstellung einer Arbeit an der TH Nürnberg
%
%%%%%%%%%%%%%%%%%%%%%%%%%%%%%%%%%%%%%%%%%%%%%%%%%%%%%%%%%%%%%%%%%%

% Definition des Grundsätzlichen dokumententyps und -layouts
\documentclass[parskip=half, titlepage=yes, 12pt, BCOR=12mm, DIV=calc]{scrartcl}

% Definition der Praeambel
% Praeambel = Quasi-Klasse mit Definitionen des Layouts
% und Definition einbinden der verwendeten Pakete
%!TeX root=Main.tex


%%%%%%%%%%%%%%%%%%%%%%%%%%%%%%%%%%%%%%%%%%%%%%%%%%%
%%%           Allgemeine Einstellungen          %%%
%%%%%%%%%%%%%%%%%%%%%%%%%%%%%%%%%%%%%%%%%%%%%%%%%%%


% Pakete für die Darstellung und Eingabemöglichkeit von Umlauten
\usepackage[T1]{fontenc}
\usepackage[utf8]{inputenc}


% Schriftarten anpassen auf helvet für seriefenlose Schrift 
% und passende Mathematik-Schrift
\usepackage{mathptmx}
\usepackage{courier}
\usepackage{microtype}     % Paket für Randausgleich

% Paket zum verwendeten erweiterter Floatoptionen
\usepackage{float}

% Pakete zum Einbinden von Bildern und Farben
\usepackage{xcolor}
\usepackage{graphicx}           


% Verwendung englischer Begriffe für automatisch generierte Worte
% wie z.B. "Contents" anstelle von "Inhaltsverzeichnis"
\usepackage[english]{babel}

% Paket um Dummy-Text/Blindtext zu erzeugen
\usepackage{blindtext}


% Mathematik Befehle und Formelsätze
\usepackage{amsmath} 
\usepackage{amssymb}
\usepackage{amsthm}
\usepackage[free-standing-units,locale = DE]{siunitx} % SI-Einheiten
\usepackage[english]{algorithm2e}     % Algorithmen
%\usepackage{MnSymbol}                % große Klammern -> falsch benutzt


% Tabellen
\usepackage{booktabs}     % Schöne Tabellen
\usepackage{supertabular} % Tabellen über mehr als eine Seite

%%%%% Literaturverzeichnis


% Literaturverzeichnis mitt BibLaTeX und Biber erstellen
\usepackage[style=numeric, sorting=none, backend=biber]{biblatex}
\addbibresource{Literature.bib}



%%%%%%%%%%%%%%%%%%%%%%%%%%%%%%%%%%%%%%%%%%%%%%%%%%%
%%%            Dokumentenlayout                 %%%
%%%%%%%%%%%%%%%%%%%%%%%%%%%%%%%%%%%%%%%%%%%%%%%%%%%


% Eigene Style-Klasse fuer eigenes Titelblatt
\usepackage{Titel}


% Allgemeine Papiergometrie
\usepackage{geometry}
\geometry{a4paper}


% Abstand der weißen Ränder
\geometry{margin=3.0cm, inner=2.5cm, outer=2.5cm} 


% Zeilenabstand auf 1,5 setzen
\usepackage[onehalfspacing]{setspace}
    \AfterTOCHead{\singlespacing}
% KOMA Klasse für europ. Layout - Wird hier nicht verwendet
  %  \KOMAoptions{DIV=last}     


% Zeit und Datumsbefehle  
\usepackage{scrdate, scrtime}   


% Erweitere Layout-Optionen
\usepackage{scrlayer-scrpage}   


% Zusätzliches Seitenlayout selbst definieren
\pagestyle{scrheadings}         


%%%%% Kopf- und Fußzeile

% Belegung von KOMA Variablen für Kopf- und Fußzeilen Gestaltung
\newcommand{\footlinetext}{\footnotesize \textsf{\color{gray}  Martin Michel, B-AMP 6  \normalsize}}

\KOMAoptions{headsepline = no, footsepline = yes}
\ihead{\headmark}
\chead{}
\ohead{}
\ifoot{\footlinetext}
\cfoot{\color{gray}\pagemark}
\ofoot{\footnotesize \textsf{\color{gray}  Radio frequency ablation with finite elements \normalsize}}



%%%%%%%%%%%%%%%%%%%%%%%%%%%%%%%%%%%%%%%%%%%%%%%%%%%
%%%   Einstellungen für Programmierquellcode    %%%
%%%%%%%%%%%%%%%%%%%%%%%%%%%%%%%%%%%%%%%%%%%%%%%%%%%


% Paket und Spezifikation der Parameter zur Darstellung 
% von Programmcode mit Courier-Schriftart
\usepackage{listings}  % Darstellung von Quellcode
\usepackage{courier}   % Schriftart laden


% Anzeigeeinstellungen für C++ Code
% Siehe Dokumentation für weitere Sprachen
\lstset{
    language=C++,
    basicstyle=\footnotesize\ttfamily,  % Standardschrift
    numbers=left,                       % Ort der Zeilennummern
    numberstyle=\tiny,                  % Stil der Zeilennummern
    %stepnumber=2,  % Abstand zwischen den Zeilennummern
    numbersep=5pt,  % Abstand der Nummern zum Text
    tabsize=2,      % Groesse von Tabs
    extendedchars=true, %
    breaklines=true,    % Zeilen werden Umgebrochen
    keywordstyle=\color{blue}\bfseries,
    frame=b,
    % keywordstyle=[1]\textbf, % Stil der Keywords
    % keywordstyle=[2]\textbf, %
    % keywordstyle=[3]\textbf, %
    % keywordstyle=[4]\textbf, \sqrt{\sqrt{}} %
    stringstyle=\color{magenta}\ttfamily, % Farbe der Strings
    showspaces=false,   % Keine Leerzeichen anzeigen 
    showtabs=false,     % Keine Tabs anzeigen 
    % xleftmargin=17pt,  % Abstände
    % framexleftmargin=17pt,
    % framexrightmargin=5pt,
    % framexbottommargin=4pt,
    commentstyle=\color{green!100!blue}\bfseries,
    %backgroundcolor=\color{grey},
    showstringspaces=true,      % Leerzeichen in Strings anzeigen
    morekeywords={__global__},  % additional language specific keywords
    morecomment=[l]
    }

    
% Syntax-Highligthing aktivieren
\lstloadlanguages{ 
    %[Visual]Basic
    %Pascal
    C,
    C++,
    %XML
    %HTML
    Matlab
    %Java
}



%%%%%%%%%%%%%%%%%%%%%%%%%%%%%%%%%%%%%%%%%%%%%%%%%%%
%%%           Layout im PDF Viewer              %%%
%%%    Muss als letztes Paket geladen werden    %%%
%%%%%%%%%%%%%%%%%%%%%%%%%%%%%%%%%%%%%%%%%%%%%%%%%%%


\usepackage{hyperref}           
\hypersetup{
    plainpages=false,
    linktocpage=true,
    breaklinks=true,
    colorlinks=true,
    linkcolor=black,%blue,
    anchorcolor=black,
    citecolor=black,%green,
    filecolor=black,%blue,
    urlcolor=black,%blue%
    pdfstartview={FitV},
    pdfview={FitH},
    pdfpagelayout={SinglePage},
    pdfpagemode={None},
}




%%%%%%%%%%%%%%%%%%%%%%%%%%%%%%%%%%%%%%%%%%%%%%%%%%%
%%%          Variablen zum Belegen              %%%
%%%%%%%%%%%%%%%%%%%%%%%%%%%%%%%%%%%%%%%%%%%%%%%%%%%


%%%%%% Variablen für Titelblatt

\DAAutor{Martin Michel}
\DATyp{Report of the application project}
\DAAutorAdresse{Keßlerplatz 12\par DE-90489 Nuremberg}
\DAFachbereich{AMP}
\DATitel{Simulation of a medical therapy \\ method with finite elements}
\DABetreuerA{Prof.\ Dr.\ rer.\ nat.\ Tim\ Kröger}
\DABetreuerB{Prof.\ Dr.\ rer.\ nat.\ habil.\ Jörg\ Steinbach}
\DABetreuerC{Prof.\ Dr.\ rer.\ nat.\ Thomas\ Lauterbach}
\DAOrt{Nuremberg}
\DAAbgabedatum{31.\,August\,2020}


\usepackage{amsthm}
\usepackage{lipsum}

%% Hier startet der eigentliche Dokumenteninhalt
\begin{document}


%% Titelblatt und Verzeichnisse
\maketitle
\tableofcontents
%\clearpage
\listoffigures
\listoftables
\lstlistoflistings
\clearpage


% part 1, Introduction 

\section{Introduction}

\textcolor{blue}{Lets talk about:\\
- Medical Treatment of Tumor\\
- Radio frequency ablation\\
- Why RFA Simulation is important \\
- Motivation \\
- This project in General\\
}

% part 2, Main

\section{Computer-aided simulation of radio frequency ablation with finite elements}

\textcolor{blue}
{
Introduction
}

\subsection{About Errors in numerical appoaches}
- see TUM dissertation \\

- There are different sources for errors following the simulation from the line from the real problem down to the discrete solution \\

- Idealization error: discrepancy between reality and the idealized reality and the idealized constitutive laws and boundary conditions -> Systems are often way more complex in reality, every patient is different \\

- Modeling errors: discrepancy between mathematical formulation and physical model -> e.g. using dimensionally reduced approaches, like linear dependencies or even constant parameters  \\

-  Discretization errors: discrepancy between the continous description and discrete discription of the model \\

- Solution errors: using iterative approximation methods and rounding errors \\

- It's basically a butterfly effect \\

- Optimizing one error source often conflicts with another one -> e.g. handling nonlinearity can cause fatal numerical errors (at least that's what Kroeger said ...)

\subsection{Theory of finite elements}

- Elliptical problems \\
- Using the cylindric domain \\
- Axial symmetrie 
- Torus elements

- Rewrite the equations to zylindric coordinates \\
 

\subsection{This part is about the concrete PDEs itself}


Laplace in kartesian coordinates:
\begin{equation}
    \nabla^2 := \Delta := \frac{\partial^2}{\partial x^2} + \frac{\partial^2}{\partial y^2} + \frac{\partial^2}{\partial z^2}
\end{equation}

Laplace in cylindric coordinates:
\begin{equation}
    \Delta := \frac{\partial^2}{\partial r^2} + \frac{1}{r} \frac{\partial}{\partial r} + \frac{1}{r^2} \frac{\partial^2 f}{\partial \varphi^2} + \frac{\partial^2 f}{\partial z^2}
\end{equation}

Laplace in polar coordinates:
\begin{equation}
    \Delta := \frac{\partial^2}{\partial r^2} + \frac{1}{r} \frac{\partial}{\partial r} + \frac{1}{r^2} \frac{\partial^2}{\partial \varphi^2}
\end{equation}




\subsection{PDE for Electric potential}

Three parts are interesting:
- Inner domain \\
- Fixed Potential of electrodes \\
- Inner domain \\
- Outer boundary -> Robin \\

Constant material parameters: \\

\subsubsection{Inner Domain}

The PDE : 

\begin{equation}
    - \nabla \cdot (\sigma(x,y,z,t) \nabla \varphi(x,y,z,t)) = 0
\end{equation}

- Elliptical boundary problem \\
- Assuming constant material parameters: Equation becomes Laplaces equation, phi becomes time independant

\begin{equation}
    \sigma \Delta \phi(x,y,z) = 0
\end{equation}

- Using zylindric domain, we can use cylindric coordinates



\subsubsection{Electrodes}

\subsubsection{Outer boundary}

The inner domain: 
\begin{equation}
    
\end{equation}



\section{Applied FEM-Simulation}

\subsection{Grid generation / Triangulation}

\subsection{Graphical output}

\subsection{Numerical challenges}

\subsection{Optimization}

\subsection{MatLab vs C++}

- Basically the performance advantages of using C++

\subsection{Interpretation of numerical solutions}

% part 3, Summary

\section{Summary and Outlook}

\subsection{Project Summary}


\textcolor{blue}
{
This is the conclusion part
}

\subsection{Current Research}
- Research in the simulation of medical therapy methods \\

\subsection{Other FEM projects and software}


\newpage


%% Literaturverzeichnis mit allen Einträgen der .bib-Datei
\clearpage
\nocite{*}
\printbibliography

\newpage

%% Letzter Teil, Anhang
\appendix

%!TeX root=Main.tex

% Dieses File ist für den Anhang
% Hier kann potenzieller Quellcode generiert werden
% Sowie zusätzliche Bilder usw eingefügt werden
% Wird nicht für alle Versuche benötigt werden

\section{Source code Visual C++}

\lstset{language=C++,
                basicstyle=\ttfamily,
                keywordstyle=\color{blue}\ttfamily,
                stringstyle=\color{red}\ttfamily,
                commentstyle=\color{gray}\ttfamily,
                morecomment=[l][\color{orange}]{\#}
}

\begin{lstlisting}[caption={[Demo] For loop to print numbers from 1 to 10}]
// Print numbers from 1 to 10
#include <stdio.h>
int main() {
  int i;
  for (i = 1; i < 11; ++i)
  {
    printf("%d ", i);
  }
  return 0;
}
\end{lstlisting}


\section{Source code MatLab}

\Large{TODO}
 

\end{document} 